\documentclass[../../../songbook.tex]{subfiles}
\begin{document}
\TabPositions{8cm} %indywidualnie dla piosenki
\section*{KRAMU TEGO KRÓL}
\phantomsection  
\addcontentsline{toc}{section}{Musicale - Kramu Tego Król}
\rightline{(Musicale-"Les Miserables")} 
\vspace{0.5cm}

{\tiny
Witam, M’sieur', wejdź pod mój dach		\tab    {\color{red}\textbf{a} } \newline
Bo oberżysty nie masz jak ja			\tab    {\color{red}\textbf{a E7} } \newline
Karczmy tu znam, jadło w nich też		\tab    {\color{red}\textbf{E7} } \newline
Ceny złodziejskie - nie wiesz co jesz	\tab    {\color{red}\textbf{E7 a } } \newline
Ja zasady mam – klient to mój pan		\tab    {\color{red}\textbf{d a} } \newline
Uczciwości wzór, mówią o mnie tak:		\tab    {\color{red}\textbf{H7 E/fis/E7} } \newline

Kramu tego król, szynkarz, że ho-ho!					\tab    {\color{red}\textbf{A} } \newline
Tu uścisnę rękę, tu nadstawię dłoń						\tab    {\color{red}\textbf{A} } \newline
Powiem sprośny żart, plotkę z wyższych sfer,			\tab    {\color{red}\textbf{A} } \newline
Goście lubią serwis w stylu bon viveur.					\tab    {\color{red}\textbf{A H7} } \newline
Tu przysługa, tam obsługa. Lecz obsługa cenę ma			\tab    {\color{red}\textbf{E} } \newline
Śpiewka to niedługa – Chcesz być gościem? No to płać!	\tab    {\color{red}\textbf{E A} } \newline

Kramu tego król jak dozorca ZOO							\tab    {\color{red}\textbf{A} } \newline
Wierny menażerii po ostatni grosz						\tab    {\color{red}\textbf{A} } \newline
Napij się za dwóch, zapłać mi za trzech.				\tab    {\color{red}\textbf{A} } \newline
„Liczę, a więc jestem” – tak filozof rzekł.				\tab    {\color{red}\textbf{A H7} } \newline
Każdy pijak mnie uwielbia, kiedy zmąci mu się wzrok		\tab    {\color{red}\textbf{E Cis fis} } \newline
I komu oni wierzą – Jezu! Przecież goli wyjdą stąd!		\tab    {\color{red}\textbf{D E7 A} } \newline

\-\hspace{1cm} Kramu tego król, szynku tego szejk							\tab    {\color{red}\textbf{A} } \newline
\-\hspace{1cm} Czysty oberżysta dziś ugości cię							\tab    {\color{red}\textbf{A} } \newline
\-\hspace{1cm} Wino leje nam z wodą pół na pół								\tab    {\color{red}\textbf{A} } \newline
\-\hspace{1cm} Więcej nie potrzeba, aby wpaść pod stół						\tab    {\color{red}\textbf{A H7} } \newline
\-\hspace{1cm} Każdy widzi w nim kompana, każdy jest z nim za pan brat		\tab    {\color{red}\textbf{E Cis  fis} } \newline
\-\hspace{1cm} A oni nic nie wiedzą – Jezu! Toć oskubię ich do cna			\tab    {\color{red}\textbf{D E7 A } } \newline

Proszę M’sieur, rozgość się tu,				 \newline	
Nogi wyprostuj i buty zzuj				 \newline
Bagaż to garb, ciężki, że hej!				 \newline	
Wezmę sakiewkę, a będzie ci lżej				 \newline
Już tam gęga gęś, już tam kwiczy schab				 \newline
Wszak o to co tu zjesz osobiście dbam				 \newline

Kiszka marsza gra? U mnie podje człek				 \newline		
Końska nerka w sosie przypomina stek				 \newline		
Własny wyrób więc, czasem bywa, że				 \newline		
Kot zamiauczy w brzuchu lub zaszczeka pies				 \newline	
Chętnie oferuję nocleg, po posiłku służy sen				 \newline	
Dodatkowe stawki do umiarkowanych dodam cen				 \newline

Wliczę każdą mysz, wliczę każdą wesz,				 \newline		
Jak dostaniesz świerzbu, to doliczę świerzb!			 \newline		
Stawka rośnie gdy klient zamknie drzwi				 \newline		
Plus piętnaście procent za nieczyste sny				 \newline		
Ceny u mnie okazyjne – Co okazja to wzrost cen				 \newline	
To nie żaden wyzysk, mamy w końcu kryzys				 \newline	
Jezu czy ty widzisz jak mi źle!					 \newline		

\-\hspace{1cm} Kramu tego król, szynku tego szejk			 \newline			
\-\hspace{1cm} Czysty oberżysta dziś ugości cię				 \newline			
\-\hspace{1cm} Wino leje nam z wodą pół na pół				 \newline		
\-\hspace{1cm} Więcej nie potrzeba by nas zwalić z nóg				 \newline		
\-\hspace{1cm} Każdy widzi w nim kamrata, każdy chce z nim napić się				 \newline
\-\hspace{1cm} Same tałatajstwo, Jezu! To pijaństwo zniszczy mnie!			 \newline

Marzyłam, że bajkowy wezmę ślub,				\tab    {\color{red}\textbf{e} } \newline
Lecz Bóg Wszechmocny nie wysłuchał moich próśb	\tab    {\color{red}\textbf{a H} } \newline

„Kramu tego król”? Ja tam swoje wiem!					\tab    {\color{red}\textbf{E} } \newline
„Czysty oberżysta” czy Podpity wieprz?					\tab    {\color{red}\textbf{E} } \newline
Jaki tęgi łeb, prawie jak Voltaire!						\tab    {\color{red}\textbf{E} } \newline
Szkoda, że w sypialni taki cienki jest					\tab    {\color{red}\textbf{E Fis7} } \newline
To okrutny wybryk losu, że takiego męża mam				\tab    {\color{red}\textbf{H Gis cis} } \newline
Gdzie ja oczy miałam, jak się wpakowałam w taki kram?!	\tab    {\color{red}\textbf{A H E} } \newline

Kramu tego król! To dopiero żart!		\tab    {\color{red}\textbf{A} } \newline
Czysty oberżysta nasz. A, psia jego mać!		\tab    {\color{red}\textbf{A} } \newline
Wino leje nam z wodą pół na pół			\tab    {\color{red}\textbf{A} } \newline
Stary hipokryta umie dbać o stół!		\tab    {\color{red}\textbf{A H7} } \newline

Niech nam żyje nasz gospodarz! Gospodyni – żyj sto lat!		\tab    {\color{red}\textbf{E Cis  fis} } \newline
Komu w drogę, temu chlup! W oberżysty głupi dziób!			\tab    {\color{red}\textbf{D E7 D E7} } \newline
Niechaj żyją nam sto lat! Wiwat kramu tego król! 			\tab    {\color{red}\textbf{D E7 A D – A D A D A E A} } \newline
}
\end{document}
