\documentclass[../../../songbook.tex]{subfiles}
\begin{document}
\TabPositions{8cm} %indywidualnie dla piosenki
\section*{KOŃ NA BIAŁYM RYCERZU}
\phantomsection  
\addcontentsline{toc}{section}{Nocny Kochanek - Koń Na Białym Rycerzu}
\rightline{(Nocny Kochanek-"Randka W Ciemność")} 
\vspace{0.5cm}
{\color{red}\textbf{d C d } } \newline
{\color{red}\textbf{d C d } } \newline
{\color{red}\textbf{B F g A } } \newline

Samotna w wieży zamknięta	\tab    {\color{red}\textbf{d C} } \newline
Z głodu przestała już jeść	\tab    {\color{red}\textbf{d C} } \newline
Między udami, gdzie kiedyś wiatr hulał,	\tab    {\color{red}\textbf{B F} } \newline
Rozciąga się pajęcza sieć	\tab    {\color{red}\textbf{g A} } \newline
Minęło dokładnie lat pięćdziesiąt parę	\tab    {\color{red}\textbf{d C} } \newline
A może nawet i mniej	\tab    {\color{red}\textbf{d C} } \newline
I choć się niewielu starało	\tab    {\color{red}\textbf{B F} } \newline
To żaden nie uwolnił jej	\tab    {\color{red}\textbf{g A d} } \newline

Wierzy głęboko, że zobaczy w wieży. \newline
Tego, o którym wciąż śni. \newline
Wie, że się rzuci na niego jak zwierzę, \newline
Gdy ten przerwie pajęczą nić. \newline
Dziwne stękanie się nagle rozlega. \newline
Czyżby to ktoś wspinał się? \newline
Przez okno księżniczka dostrzega. \newline
Konia co rży, czy tam rżnie. \newline

\-\hspace{1cm} To był koń na białym rycerzu	\tab    {\color{red}\textbf{d C d} } \newline
\-\hspace{1cm} Na białym rycerzu był koń.	\tab    {\color{red}\textbf{d C} } \newline
\-\hspace{1cm} To był koń na białym rycerzu	\tab    {\color{red}\textbf{B F} } \newline
\-\hspace{1cm} Na białym rycerzu był koń	\tab    {\color{red}\textbf{g A} } \newline

\-\hspace{1cm} To był koń na białym rycerzu \newline
\-\hspace{1cm} Na białym rycerzu był koń \newline
\-\hspace{1cm} To był koń na białym rycerzu \newline
\-\hspace{1cm} Na białym rycerzu był koń	\tab    {\color{red}\textbf{B A d} } \newline
\end{document}
