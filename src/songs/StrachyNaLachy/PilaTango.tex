\documentclass[../../../songbook.tex]{subfiles}
\begin{document}
\TabPositions{8cm} %indywidualnie dla piosenki
\section*{PIŁA TANGO}
\phantomsection  
\addcontentsline{toc}{section}{Strachy Na Lachy - Piła Tango}
\rightline{(Strachy Na Lachy-"Piła Tango")} 
\vspace{0.5cm}
\setlength{\columnsep}{0.1cm}
\begin{multicols}{2}
[\textbf{Cała piosenka: {\color{red}  a Asus2 a d E  }} 
]
{\footnotesize
Oto historia z kantem,	\newline
Co podwójne ma dno.	\newline
Gdyby napisał ją Dante,	\newline
To nie tak by to szło...	\newline

Grzesiek Kubiak, czyli Kuba 	\newline
Rządził naszą podstawówką;	\newline
Po lekcjach na boisku 	\newline
Ganiał za mną z cegłówką.	\newline
W Pile było jak w Chile,	\newline
Każdy miał czerwone ryło,	\newline
Mniej lub bardziej to pamiętasz 	\newline- 
Spytaj jak to było,	\newline
W czasach gdy nad Piłą 	\newline
Jeszcze latały samoloty;	\newline
Wojewoda Śliwiński 	\newline
Kazał pomalować płoty.	\newline
Potem wszystkie płoty w Pile 	\newline
Miały kolor zieleni;	\newline
Rogaczem na wieżowcu 	\newline
Piła witała jeleni.	\newline

\-\hspace{0.5cm} Statek Piła Tango,	\newline
\-\hspace{0.5cm} Czarna bandera.	\newline
\-\hspace{0.5cm} To tylko Piła Tango;	\newline
\-\hspace{0.5cm} Tańczysz to teraz.	\newline
\-\hspace{0.5cm} Płynie statek Piła Tango,	\newline
\-\hspace{0.5cm} Czarna Bandera.	\newline
\-\hspace{0.5cm} Ukłoń się świrom,	\newline
\-\hspace{0.5cm} Żyj nie umieraj.	\newline

Gruby jak armata Szczepan 	\newline
Błąkał się po kuli ziemskiej,	\newline
Trafił do Ameryki 	\newline
Prosto z Legii Cudzoziemskiej.	\newline
Baca w Londynie z Buchami się sąsiedzi,	\newline
Lżej się tam halucynuje, 	\newline
Nikt go tam nie śledzi.	\newline
Karawan z Holandii, 	\newline
On przyjechał tutaj wreszcie,	\newline
Są już Kula, Czarny Dusioł - 	\newline
Słychać strzały na mieście.	\newline
Znam jednak takie miejsca 	\newline
Gdzie jest lepiej chodzić z nożem;	\newline
Całe Górne i Podlasie - 	\newline
Wszyscy są za Kolejorzem.	\newline

\-\hspace{0.5cm} Statek Piła Tango...	\newline

Andrzej Kozak, Mandaryn - 	\newline
Znana postać medialna;	\newline
Tyci przy nim jest kosmos, 	\newline
Gaśnie gwiazda polarna.	\newline
Jest tu Siwy, który w rękach 	\newline
Niebezpieczne ma narzędzie,	\newline
A kiedy Siwy tańczy - 	\newline
Znaczy mordobicie będzie.	\newline
U Budzików pod tytułem 	\newline
Chleją nawet z gór szkieły;	\newline
Zbigu śpi przy stoliku, 	\newline
Ma nieczynny przełyk.	\newline
Lecz spokojnie panowie, 	\newline
Według mej najlepszej wiedzy,	\newline
Najszersze gardła tu to mają z INRI koledzy...	\newline

\-\hspace{0.5cm} Statek Piła Tango...	\newline

Nad rzeką, latem ferajna na grilla się zasadza...	\newline
Auta z Niemiec? Sam wiem kto je tu sprowadza;	\newline
Żaden spleen i cud, na ulicach nie śpią złotówki,	\newline
W Pile Święta jest Rodzina i święte są żarówki.	\newline
Nic nie szkodzi, że z wieczora 	\newline
Miasto dławi się w fetorach...	\newline
Ważne że jest żużel i kiełbasy senatora!	\newline
Fajne z Wincentego Pola idą w świat dziewczyny;	\newline
Po pokładzie jeździ Jojo bicyklem z Ukrainy.	\newline

\-\hspace{0.5cm} Statek Piła Tango...	\newline

Oto historia z kantem,	\newline
Co podwójne ma dno.	\newline
Gdyby napisał ją Dante,	\newline
To nie tak by to szło...	\newline
By szło, by szło...	\newline
}
\end{multicols}
\end{document}
