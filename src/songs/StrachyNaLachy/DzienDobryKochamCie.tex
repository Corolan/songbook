\documentclass[../../../songbook.tex]{subfiles}
\begin{document}
\TabPositions{8cm} %indywidualnie dla piosenki
\section*{DZIEŃ DOBRY, KOCHAM CIĘ}
\phantomsection  
\addcontentsline{toc}{section}{Strachy Na Lachy - Dzień Dobry, Kocham Cię}
\rightline{(Strachy Na Lachy-"Piła Tango")} 
\vspace{0.5cm}
Bo chodzi o to by od siebie nie upaść za daleko 		\tab    {\color{red}\textbf{D fis} } \newline
Jak te dwa łyse kamienie nad rzeką  \newline
Chodzi o to by pierwsze chciało słuchać  \newline
Co mu to drugie powiedzieć chce do ucha:  \newline
Że po mej głowie - czasem się ich boje -  \newline
Chodzą słowa nie do powiedzenia... Nie do powiedzenia  \newline
Chodzą słowa nie do powiedzenia... Nie do powiedzenia  \newline

\-\hspace{1cm} Dzień dobry kocham cię - już posmarowałem tobą chleb 	\quad \quad    {\color{red}\textbf{D E fis} } \newline
\-\hspace{1cm} Dzień dobry kocham cię - nie chce cię z oczu stracić więc  \newline
\-\hspace{1cm} Jeszcze więcej  \newline
\-\hspace{1cm} Dzień dobry kocham cię - podzielimy dziś ten ogień na dwoje  \newline
\-\hspace{1cm} Dzień dobry kocham cię - to zapyziałe miasto niech o tym wie  \newline

Tu chodzi o to by od siebie nie upaść za daleko  \newline
Kiedy długo drugie nie widzi pierwszego  \newline
Bo gdy siedzi człek samemu z czarnymi myślami  \newline
Człowiek rzuca słuchawkami, rzuca słuchawkami  \newline

Bo chodzi o to by od siebie nie upaść za daleko  \newline
Nawet jeśli czasem między nami wykipi mleko  \newline
Choćbyś nawet i wieczorem zasypiała zdołowana  \newline
Chciałbym ci zaśpiewać z rana  \newline
Móc ci zaśpiewać z rana; kochana...  \newline

\-\hspace{1cm} Dzień dobry...  \newline

\-\hspace{1cm} Para-moje  \newline
\-\hspace{1cm} para-twoje  \newline
\-\hspace{1cm} Onomatopeiczne  \newline
\-\hspace{1cm} Paranormalne  \newline
\-\hspace{1cm} Paranoje  \newline
\-\hspace{1cm} We dwoje  \newline

\end{document}
