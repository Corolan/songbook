\documentclass[../../../songbook.tex]{subfiles}
\begin{document}
\TabPositions{8cm} %indywidualnie dla piosenki
\section*{NA POGRZEB KRÓLA}
\phantomsection  
\addcontentsline{toc}{section}{Strachy Na Lachy - Na Pogrzeb Króla}
\rightline{(Strachy Na Lachy-"Piła Tango")} 
\vspace{0.1cm}
\setlength{\columnsep}{0.1cm}
\begin{multicols}{2}
[\textbf{Cała piosenka: {\color{red}  d a  }} 
]
{\footnotesize
\textit{Zbrodnia, kara, grzech, litr wina \newline
I tak historia ta się zaczyna... \newline}

W pokoju tym po szarym dywanie \newline
Rozsypał się różaniec \newline
Zaczął się mój żywot ze strachem \newline
Odtąd mieszkam z nim tu pod jednym dachem \newline
I tak patrzysz przez tę klucza dziurkę... \newline

Miałem tam niejedną schadzkę \newline
Z niejakim Kaczmarskim Jackiem \newline
Nurzaliśmy się w życia dożynki \newline
Dłubaliśmy z ciast rodzynki \newline
Mistrzów śniadania- tak żyła nasza kompania \newline
Kokaina i crack na rusztach  \newline
aż do stu finałów w ustach \newline
Aż do dnia pewnego, wielkanocnego... \newline

I nie pomógł żaden cudowny proszek \newline
Ani wdowie cztery grosze \newline
Na czarną godzinę skryte \newline
W czarnym pasku zaszyte \newline
W innym stadium opętania \newline
Ostatnie swe sprzedał ubrania \newline
Do dziś w kasie pancernej Króla \newline
Wisi jego koszula... \newline
A wiatr jak hulał tak hula \newline
W czarnych tiulach \newline

\-\hspace{0.4cm} U Króla na dworze też coraz gorzej \newline
\-\hspace{0.4cm} Święte lampy gorzkim żalem lśnią na placach  \newline
\-\hspace{0.4cm} Paprze się ten zgorzel na tym dworze \newline
\-\hspace{0.4cm} W stosie pustych kałamarzy  \newline
\-\hspace{0.4cm} Schną diamentowe gęsie pióra \newline

\textit{Naród kefir ma na kaca \newline
Czas drugą stronę przewraca...} \newline

Do morza dusz głów wpada rzeka \newline
Jadą tu z bliska i z daleka \newline
Tam gdzie horyzontu schody \newline
Ciągną się korowody \newline
Wiezie pociąg dary dla króla \newline
Puchar Tubę i Okular \newline
Nowe są w lokomotywie koła  \newline
W tej co jeździ dookoła \newline
Jedno wolne miejsce w tylnych rzędach \newline
Dla pewnego dyrygenta \newline
Pastuszkowie mu śpiewają sto lat \newline
Czterech starców w aureolach \newline
Ciągną powietrzne sanie \newline
Pora wypić za to spotkanie \newline
I za duszy szaber \newline
Hare Kriszna, szaber, szaber \newline

\-\hspace{0.4cm} U króla na dworze coraz gorzej... \newline

\textit{Teraz pomęczymy kota nim się zacznie 3 zwrota \newline
Więc wybaczy pan i pani \newline
Krótko będziem was cyganić...} \newline

Na króla dworze po krzywych szynach \newline
Pędzi dziejów maszyna \newline
Na wielkim jak plac ekranie \newline
Ustał króla ze śmiercią taniec  \newline
I tylko mędrcy Syjonu \newline
Nie chcą mówić o tym nikomu \newline
Strach im tak zasznurował usta \newline
Kto? Kto z głośników samochodów \newline
Przemówi do narodu? \newline
Jeśli się ludu tego boi \newline
Nawet ten co nad królem stoi \newline
Komunikat bezlitośnie prosty: \newline
Pan nasz wącha wodorosty \newline
Zamknęły się Króla powieki \newline
Pan wszedł do umarłej rzeki \newline
I tak ślepisz przez tę klucza dziurkę... \newline
Patrzysz tak przez tę wąską szparę:  \newline
Ktoś z ręki zrywa mu zegarek \newline
Ściągają złoty płaszcz przez głowę \newline
Wyrywają zęby platynowe \newline
Martwy sam palec serdeczny \newline
Zgasł na palcu pierścień wieczny \newline
Widzisz ten tłok piekielny w szatni? \newline
Nikt nie chce stąd uciec ostatni \newline
Przemykają tak chyłkiem pod murem \newline
Szczur za szczurem, sępy sznurem \newline

\-\hspace{0.4cm} Gorzej być nie może  na tym dworze \newline
\-\hspace{0.4cm} Obce orły błąkają się po salach \newline
\-\hspace{0.4cm} Pękają na pół zorze \newline
\-\hspace{0.4cm} Słodki Boże mój \newline
\-\hspace{0.4cm} Ty wiesz że nie mam dokąd już stąd spierdalać \newline

U króla na dworze, słodki Boże \newline
W stosie pustych kałamarzy  \newline
Wyschły diamentowe gęsie pióra  \newline

}
\end{multicols}
\end{document}
