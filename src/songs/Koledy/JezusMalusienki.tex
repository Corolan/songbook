\documentclass[../../../songbook.tex]{subfiles}
\begin{document}
\TabPositions{8cm} %indywidualnie dla piosenki
\section*{JEZUS MALUSIEŃKI}
\phantomsection  
\addcontentsline{toc}{section}{Kolędy - Jezus Malusieńki}
\rightline{(Kolędy-"Nieznany")} 
\vspace{0.5cm}
Jezus malusieńki leży wśród stajenki, 			 \tab    {\color{red}\textbf{a E a E  } } \newline
Płacze z zimna, nie dała mu Matula sukienki. 	  \tab    {\color{red}\textbf{d C G7 C /E} } \newline

Bo uboga była,rąbek z głowy zdjęła, 		 \newline
W który dziecię owinąwszy, siankiem go okryła.  		 \newline

Nie ma kolebeczki ani poduszeczki. 		 \newline
We żłobie mu położyła siana pod główeczki.  		 \newline

Dziecina się kwili, matusieńka lili, 		 \newline
W nóżki zimno, żłóbek twardy, stajenka się chyli.  		 \newline

Matusia truchleje, serdeczne łzy leje.		 \newline
O mój Synu, wola Twoja, nie moja się dzieje. 		 \newline

Przestań płakać, proszę, bo żalu nie zniosę. 		 \newline
Dosyć go mam z męki Twojej, którą w sercu noszę.  		 \newline

Pokłon oddawajmy, Bogiem je wyznajmy. 		 \newline
To Dzieciątko ubożuchne ludziom ogłaszajmy.  		 \newline

Niech je wszyscy znają, serdecznie kochają, 		 \newline
Za tak wielkie poniżenie chwałę mu oddają.  		 \newline

O najmilszy panie, waleczny Hetmanie! 		 \newline
Zwyciężonyś mając rączki miłością związane.  		 \newline

Leżysz na tym sianie, Królu nieba, ziemi, 		 \newline
Jak baranek na zabicie za moje zbawienie. 		 \newline
\end{document}
