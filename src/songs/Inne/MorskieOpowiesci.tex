\documentclass[../../../songbook.tex]{subfiles}
\begin{document}
\TabPositions{8cm} %indywidualnie dla piosenki
\section*{MORSKIE OPOWIEŚCI}
\phantomsection  
\addcontentsline{toc}{section}{Inne - Morskie Opowieści}
\rightline{(Inne-"Nieznany")} 
\vspace{0.5cm}
\setlength{\columnsep}{0.1cm}
\begin{multicols}{3}
[\textbf{Cała piosenka: {\color{red}  a - G - a - G E7 a  }} 
]
{\tiny
Kiedy rum zaszumi w głowie,  \newline
Cały świat nabiera treści,  \newline
Wtedy chętniej słucha człowiek  \newline
Morskich opowieści.  \newline

Hej, ha! Kolejkę nalej! \newline
Hej, ha! Kielichy wznieśmy! \newline
To zrobi doskonale \newline
Morskim opowieściom. \newline

Niech drżą gitary struny, \newline
Niech wiatr grzywacze pieści, \newline
Gdy płyniemy pod banderą \newline
Morskich opowieści. \newline

Może ktoś się będzie zżymał \newline
Mówiąc, że to zdrożne wieści, \newline
Ale to jest właśnie klimat \newline
Morskich opowieści. \newline

Łajba to jest morski statek, \newline
Sztorm to wiatr co dmucha z gestem, \newline
Cierpi kraj na niedostatek \newline
Morskich opowieści. \newline

Pływał raz marynarz, który \newline
Żywił się wyłącznie pieprzem, \newline
Sypał pieprz do konfitury \newline
I do zupy mlecznej. \newline

Był na "Lwowie" młodszy majtek, \newline
Czort, Rasputin, bestia taka, \newline
Że sam kręcił kabestanem \newline
I to bez handszpaka. \newline

Jak pod Helem raz dmuchnęło, \newline
Żagle zdarła moc nadludzka, \newline
Patrzę - w koję mi przywiało \newline
Nagą babkę z Pucka. \newline

Od Falklandu-śmy płynęli, \newline
Doskonale brała ryba, \newline
Mogłeś wędką wtedy złapać \newline
Nawet wieloryba. \newline

Rudy Joe, kiedy popił, \newline
Robił bardzo głupie miny, \newline
Albo skakał też do wody \newline
I gonił rekiny.

I choć rekin twarda sztuka, \newline
Ale Joe w wielkiej złości \newline
Łapał gada od ogona \newline
I mu łamał kości. \newline

Raz bosmana rekin pożarł, \newline
Lecz nie smućcie się kochani, \newline
Bosman żyje, rekin umarł, \newline
Zatruł się zbukami. \newline

Znałem kiedyś Chinkę w barze, \newline
Co śpiewała piosnki sprośne, \newline
Gdy kimono swe rozdziała, \newline
Cycki miała skośne. \newline

Pływał raz marynarz, który \newline
Żywił się wyłącznie wódką \newline
Dawał wódkę małolatom \newline
No i prostytutkom. \newline

Pływał z nami raz szantymen, \newline
Śpiewał bardzo niskim basem, \newline
W rękach zawsze miał gitarę, \newline
Ster trzymał... rękami. \newline

Znałem raz murzynkę w Rio, \newline
Co w miłości była śmiała, \newline
Nie uwierzysz daję słowo, \newline
Całkiem w poprzek miała. \newline

Kolumb odkrył Amerykę, \newline
Kiedy ścigał się z Halikiem, \newline
Indianie się zarzekali, \newline
Że pierwszy był Halik. \newline

O wyprawie wokół globu, \newline
Też fałszywe są pogłoski, \newline
Pierwszy żaden tam Magellan, \newline
Tylko Baranowski. \newline

Nelson, angielski Admirał, \newline
Strzeliłby se w łeb i kwita, \newline
Gdyby wiedział co dokonał, \newline
Kloss, zwykły Kapitan. \newline

Żyła w Gdańsku cnotka Zocha, \newline
Z każdym chciałaby się kochać, \newline
Lecz stalową cnotę miała, \newline
Rzewnie więc płakała. \newline

Zośka dzięki swym przymiotom, \newline
Podpuszczalska była wielce, \newline
Wielu więc miało złamane, \newline
Niekoniecznie serce. \newline

Larsen choć był harpunnikiem, \newline
Nie mógł Zośce przebić cnoty, \newline
Chociaż raz rzutem harpuna, \newline
Przebił trzy U-Booty. \newline

Grant Kapitan z żoną pływał, \newline
Nie dopatrzył raz załogi, \newline
Odtąd ma bachorów kupę, \newline
A na głowie rogi. \newline

Słuchaj rady młody majtku, \newline
Strzeż się dziewcząt w Yokohamie, \newline
Tam są gejsze takie szybkie, \newline
Zgwałcą nim ci stanie. \newline

Kiedy Bosman trypra złapał, \newline
Obciął sobie własnym nożem, \newline
A gdy rzucił go za burtę, \newline
To wezbrało morze. \newline

Mały John z Krzywym Pyskiem, \newline
Na "Darze Pomorza" pływał, \newline
A że krzywy miał interes, \newline
Pysk se obsikiwał. \newline

Kiedy znudzą ci się szanty \newline
I żegluga, i Mazury, \newline
To pierdolnij kapitana \newline
I uciekaj w góry. \newline

Powiedziała mi dziewczyna, \newline
Żeby wodą wódkę popić. \newline
A ja na to: "Idź do diabła, \newline
Czy chcesz mnie utopić". \newline

Znałem kiedyś pannę śliczną. \newline
Maszty stawiać uwielbiała, \newline
Chłopa z łajbą pomyliła, \newline
Lecz nie żałowała. \newline

Kumpel nazwać swoją łajbę \newline
Chciał tytułem jakiejś pieśni, \newline
Ja mu na to - daj jej imię \newline
"Morskie opowieści". \newline

Pływał raz marynarz, który \newline
chuja miał jak trzy armaty \newline
i wystrzałem z tej giwery \newline
zatapiał fregaty \newline

Kiedy szliśmy przez Pacyfik \newline
była wtedy straszna flauta \newline
wprost na łajbę nam się zjebał \newline
ruski kosmonauta \newline

Znałem kiedyś marynarza, \newline
kochał piwo no i tańce \newline
jak się odlał to wypełniał \newline
śluzę na Guziance \newline

Raz stanąłem w Mikołajkach \newline
patrzę, a tu z pod "Pagaja" \newline
wychodzi stary marynarz \newline
bez lewego jaja \newline

Do Giżycka dziś płyniemy \newline
nieźle daje, szóstka wieje \newline
jak tak dalej dobrze pójdzie \newline
rozpierdoli keję \newline

W dawnych czasach na okrętach \newline
żyły kozy tresowane \newline
co w rzemiośle zastąpiły \newline
każdą kurtyzanę \newline

A gdy kozy szły do kotła \newline
bo czasami tak się zdarza \newline
to wtedy cała załoga \newline
jebała kucharza \newline

Kiedy szliśmy przez Pacyfik \newline
wiatr pozrywał wszystkie wanty \newline
przytuliłem się do klopa \newline
i śpiewałem szanty \newline

Znałem raz pewnego majtka \newline
nazywaliśmy go Pszczółka \newline
jebał wszystko prócz zegarka \newline
chyba, że z kukułką \newline

Kiedy ci na rejsie smutno \newline
chcesz rozerwać się troszeczkę \newline
wsadź se granat między nogi \newline
wyciągnij zawleczkę \newline

Pewien majtek miał dwie nogi \newline
co się nie trzymały kupy \newline
bo przed laty zbił majątek \newline
na dawaniu dupy \newline

Pływał raz po morzu kucharz \newline
w rękach praktyk był onana \newline
a załoga się dziwiła \newline
skąd w kawie śmietana \newline

Pływał raz marynarz który \newline
myślał, że go dupa boli \newline
patrzy, a tu sam kapitan \newline
w koi go pierdoli \newline

Raz na wodach Adriatyku \newline
patrzym płynie rower wodny \newline
jak w niego przypierdolimy \newline
to będzie podwodny \newline

Pij bracie, pij na zdrowie \newline
Jutro Ci się humor przyda \newline
Spirytus Ci nie zaszkodzi \newline
Sztorm idzie - wyrzygasz \newline
}
\end{multicols}
\end{document}
