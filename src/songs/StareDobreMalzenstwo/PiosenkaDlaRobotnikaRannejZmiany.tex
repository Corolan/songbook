\documentclass[../../../songbook.tex]{subfiles}
\begin{document}
\TabPositions{8cm} %indywidualnie dla piosenki
\section*{PIOSENKA DLA ROBOTNIKA RANNEJ ZMIANY}
\phantomsection  
\addcontentsline{toc}{section}{Stare Dobre Małżeństwo - Piosenka Dla Robotnika Rannej Zmiany}
\rightline{(Stare Dobre Małżeństwo-"Niebieska Tancbuda")} 
\vspace{0.5cm}
Godzina słynna: piąta pięć 				\tab    {\color{red}\textbf{E } } \newline
Naciska budzik, dźwiga się \newline
Do kuchni drogę zna na pamięć \newline
Prowadzą go tam nogi same 				\tab    {\color{red}\textbf{E  E7} } \newline
Pod kran pakuje śpiący łeb 				\tab    {\color{red}\textbf{A} } \newline
Przez chwilę jeszcze śpi jak w łóżku	\tab    {\color{red}\textbf{E} } \newline
Dopóki nie posłyszy plusku 				\tab    {\color{red}\textbf{H7} } \newline
I wtedy wreszcie budzi się 				\tab    {\color{red}\textbf{A E H7} } \newline

Aniele Pracy – stróżu mój \newline
Jak ciężki robotnika znój \newline
Zbożowa kawa, smalec, chleb \newline
Salceson czasem, kiedy jest \newline
Do teczki drugie pcha śniadanie \newline
I teraz szybko na przystanek \newline
W tramwaju tłok i nie ma Boga \newline
Jest ramię w ramię, w nogę noga \newline
Kimanie na stojąco jest \newline

Aniele Pracy – stróżu mój \newline
Jak ciężki robotnika znój \newline
Przez osiem godzin praca wre \newline
Jak z bicza strzelił minął dzień \newline
Już w domu siedzi przed ekranem \newline
Na stole flaszka z marcepanem \newline
Dziś chłopcy grają ważny mecz \newline
Przez cały czas w ataku nasi \newline

Aniele Pracy – stróżu mój \newline
Jak ciężki robotnika znój \newline
Nich nas ukoi dobry sen \newline
Najlepsza w końcu jest to rzecz \newline
I co się śni? Podwyżka cen \newline
Aniele Pracy – stróżu mój \newline
Jak ciężki robotnika znój \newline
\end{document}
