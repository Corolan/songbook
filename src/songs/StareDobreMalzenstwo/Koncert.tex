\documentclass[../../../songbook.tex]{subfiles}
\begin{document}
\TabPositions{9cm} %indywidualnie dla piosenki
\section*{KONCERT}
\phantomsection  
\addcontentsline{toc}{section}{Stare Dobre Małżeństwo - Koncert}
\rightline{(Stare Dobre Małżeństwo-"Pod Wielkim Dachem Nieba")} 
\vspace{0.5cm}
W kołnierz wtulam twarz,\tab    {\color{red}\textbf{H7 e} } \newline	 
chowam się przed miastem\tab    {\color{red}\textbf{H7 e} } \newline	 
- jego cienie żłobią w mojej twarzy wąwóz.\tab    {\color{red}\textbf{C G H7} } \newline
Trzeszczy, jak ułamek szkła\tab    {\color{red}\textbf{C G } } \newline
mój codzienny niepokój\tab    {\color{red}\textbf{a G } } \newline
jak wydostać się z cienia?\tab    {\color{red}\textbf{C G } } \newline
Może wtedy				\tab    {\color{red}\textbf{H7 } } \newline

\-\hspace{1cm} Gdyby koncert grać - ten na trąbki i skrzypce\tab    {\color{red}\textbf{C G} } \newline
\-\hspace{1cm} Tak, by dźwięki ułożyły się w wiersz?		\tab    {\color{red}\textbf{C G } } \newline
\-\hspace{1cm} Gdyby łyżką światła rozweselić to wszystko	\tab    {\color{red}\textbf{C G } } \newline
\-\hspace{1cm} Żeby we mnie zaśpiewało coś też				\tab    {\color{red}\textbf{H7 e} } \newline

Rośnie we mnie mgła, \newline
jak ze studzien stu. \newline
Nie wiem, ilu trzeba ksiąg, by ją rozwiać  \newline
Jedno wiem, że muszę biec \newline
póki sił mi wystarczy, \newline
póki tylko ta nuta \newline
- mam ją w sobie! \newline

\-\hspace{1cm} Będę koncert grać - ten na trąbki i skrzypce \newline
\-\hspace{1cm} Tak, by dźwięki ułożyły się w wiersz. \newline
\-\hspace{1cm} Będę łyżką światła rozweselać to wszystko \newline
\-\hspace{1cm} żeby w tobie zaśpiewało coś też. \newline

\end{document}
