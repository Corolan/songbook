\documentclass[../../../songbook.tex]{subfiles}
\begin{document}
\TabPositions{10cm} %indywidualnie dla piosenki
\section*{TU ASFALT SIĘ KOŃCZY}
\phantomsection  
\addcontentsline{toc}{section}{Stare Dobre Małżeństwo - Tu Asfalt Się Kończy}
\rightline{(Stare Dobre Małżeństwo-"Bieszczadzkie Anioły")} 
\vspace{0.5cm}
Tu asfalt się kończy a zaczyna blues 				\tab    {\color{red}\textbf{C} } \newline
Na płocie list gończy i po prostu luz 				\tab    {\color{red}\textbf{G C} } \newline
A termometr w oknie lekko dziś moknie  				\tab    {\color{red}\textbf{C} } \newline
I wskazuje plus – minus – chyba ze trzy stopnie		\tab    {\color{red}\textbf{G C} } \newline

Gdy Ogrodowa moczy się, to tak naprawdę nie jest źle			\tab    {\color{red}\textbf{F C G C} } \newline
Bo ta ulica ma swój styl, warto tu wpaść na kilka chwil 		\tab    {\color{red}\textbf{F C G C} } \newline
Tu Leona spotkasz czasem, kiedy wypije to zaraz płacze 			\tab    {\color{red}\textbf{F C G C} } \newline
Tu żyje jeszcze Sopel Jan, był elektrykiem, dzisiaj jest sam 	\tab    {\color{red}\textbf{F C G C} } \newline

Tu asfalt się kończy i zaczyna noc			 \newline
Na płocie list gończy stracił swoją moc			 \newline
A termometr w oknie męczy się okropnie			 \newline
I wskazuje plus – minus – może ze trzy stopnie			 \newline

Na Ogrodowej pusto dziś, chłopcy do wojska musieli iść			 \newline
Kiedyś wrócą na Ogrodową, choć może już okrężną drogą			 \newline
Bo tu zostały ich pierwsze ślady i pierwsza miłość z tamtych lat			 \newline
I te piwonie które tak szalały po ogrodach przez cały czas \tab    {\color{red}\textbf{F C G C/G C} } \newline
\end{document}
