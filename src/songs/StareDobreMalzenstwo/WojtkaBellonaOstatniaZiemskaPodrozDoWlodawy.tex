\documentclass[../../../songbook.tex]{subfiles}
\begin{document}
\TabPositions{8cm} %indywidualnie dla piosenki
\section*{WOJTKA BELLONA OSTATNIA ZIEMSKA PODRÓŻ DO WŁODAWY}
\phantomsection  
\addcontentsline{toc}{section}{Stare Dobre Małżeństwo - Wojtka Bellona Ostatnia Ziemska Podróż Do Włodawy}
\rightline{(Stare Dobre Małżeństwo-"Czarny Blues O Czwartej Nad Ranem")} 
\vspace{0.5cm}

\setlength{\columnsep}{0.1cm}
\begin{multicols}{2}
{\footnotesize
Z gitarą i piórem kwietniowym wieczorem		 {\color{red}\textbf{d C} } \newline
Jedziemy Wojtku razem do Włodawy		\quad\quad	 {\color{red}\textbf{G} } \newline
Stary bieszczadnik Majster Bieda		\quad\quad\quad	 {\color{red}\textbf{F G} } \newline
Wciąż wierny górom jak zwykle jest z nami	 {\color{red}\textbf{A A7} } \newline
	
\-\hspace{1cm} I mówisz – wszystko się uda	 {\color{red}\textbf{d F} } \newline
\-\hspace{1cm} O to nie ma żadnej obawy		\quad {\color{red}\textbf{C A} } \newline
\-\hspace{1cm} Przecież jedziemy dziś		\quad\quad {\color{red}\textbf{B C} } \newline
\-\hspace{1cm} Na dwa dni do Włodawy		\quad {\color{red}\textbf{d A d} } \newline

W przedziale po oczach mdli mleczna żarówka				\newline
Mała gwiazda betlejemska				\newline
I pociąg lubelski noc długa przecina				\newline	
Buntując się na ostrych zakrętach				\newline

\-\hspace{1cm} I mówisz – wszystko będzie dobrze 			\newline
\-\hspace{1cm} Opowiemy im nasze sprawy 			\newline
\-\hspace{1cm} Przecież po to jedziemy 			\newline
\-\hspace{1cm} Na dwa dni do Włodawy 			\newline

Z plecaka chleb wyjęty garść soli				\newline		
Kawałek sprytem zdobytej kiełbasy				\newline			
Chcemy noc przeskoczyć ciemną i niepewną				\newline	
Z biletem kupionym w nieznane					\newline	

\-\hspace{1cm} Lecz mówisz – znów musi się udać				\newline	
\-\hspace{1cm} Choć tyle podróży za nami 			\newline
\-\hspace{1cm} Przecież jedziemy dziś			\newline
\-\hspace{1cm} Na dwa dni do Włodawy 			\newline

Dzisiaj się buntuje Czesiek Król nad króle 			\newline
Piekarz rodem z Buska wchodzi w układ 			\newline
Stawia pasjans z bułek i gorzej się czuje 			\newline
A w drewnie cierpliwym został ślad 			\newline

\-\hspace{1cm} A ty na przekór wszystkim mówisz 			\newline
\-\hspace{1cm} Że się uda – nie ma obawy 			\newline
\-\hspace{1cm} Przecież jedziemy dziś			\newline
\-\hspace{1cm} Na dwa dni do Włodawy 			\newline

I są wciąż pytania i są wciąż rozmowy 			\newline
Ważne choć tylko przedziałowe 			\newline
I jak z każdej wspólnej nam posiady 			\newline
Wygląda w przyszłość zatroskany człowiek 			\newline

\-\hspace{1cm} Wyciągasz wiersz ciepły jeszcze 			\newline
\-\hspace{1cm} Wczoraj bodajże napisany 			\newline
\-\hspace{1cm} I czytasz głośno go w przedziale 			\newline
\-\hspace{1cm} W naszej podróży do Włodawy  			\newline

Z gitarą i piórem kwietniowym wieczorem					\newline
Jedziemy Wojtku razem do Włodawy					\newline	
Stary bieszczadnik Majster Bieda					\newline	
Wciąż wierny górom jak zwykle jest z nami				\newline
	
\-\hspace{1cm} I mówisz – wszystko się uda				\newline
\-\hspace{1cm} O to nie ma żadnej obawy			\newline
\-\hspace{1cm} Przecież jedziemy dziś				\newline
\-\hspace{1cm} Na dwa dni do Włodawy				\newline
	
}
\end{multicols}
\end{document}
