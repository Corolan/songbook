\documentclass[../../../songbook.tex]{subfiles}
\begin{document}
\TabPositions{9cm} %indywidualnie dla piosenki
\section*{JAK}
\phantomsection  
\addcontentsline{toc}{section}{Stare Dobre Małżeństwo - Jak}
\rightline{(Stare Dobre Małżeństwo-"Dla Wszystkich Starczy Miejsca")} 
\vspace{0.5cm}
Jak po nocnym niebie sunące białe obłoki nad lasem		\tab    {\color{red}\textbf{D A G D} } \newline		
Jak na szyi wędrowca apaszka szamotana wiatrem			\tab    {\color{red}\textbf{e G D} } \newline	
Jak wyciągnięte tam powyżej gwiaździste ramiona wasze	 \newline	
A tu są nasze, a tu są nasze.			 \newline	

\-\hspace{1cm} Jak suchy szloch w tę dżdżystą noc			 \newline
\-\hspace{1cm} Jak winny - li - niewinny sumienia wyrzut,			 \newline
\-\hspace{1cm} Że się żyje, gdy umarło tylu, tylu, tylu.			 \newline

Jak suchy szloch w tę dżdżystą noc			 \newline
Jak lizać rany celnie zadane			 \newline
Jak lepić serce w proch potrzaskane			 \newline

\-\hspace{1cm} Jak suchy szloch w tę dżdżystą noc			 \newline
\-\hspace{1cm} Pudowy kamień, pudowy kamień			 \newline
\-\hspace{1cm} Jak na nim stanę, on na mnie stanie			 \newline
\-\hspace{1cm} On na mnie stanie, spod niego wstanę			 \newline

Jak suchy szloch w tę dżdżystą noc			 \newline
Jak złota kula nad wodami			 \newline
Jak świt pod spuchniętymi powiekami			 \newline

\-\hspace{1cm} Jak zorze miłe, śliczne polany			 \newline
\-\hspace{1cm} Jak słońca pierś, jak garb swój nieść			 \newline
\-\hspace{1cm} Jak do was, siostry mgławicowe, ten zawodzący śpiew			 \newline

Jak biec do końca, potem odpoczniesz, 			 \newline
Potem odpoczniesz, cudne manowce, 			 \newline
Cudne manowce, cudne, cudne manowce			 \newline
\end{document}
