\documentclass[../../../songbook.tex]{subfiles}
\begin{document}
\TabPositions{8cm} %indywidualnie dla piosenki
\section*{WIERSZ KSIĘŻYCOWY}
\phantomsection  
\addcontentsline{toc}{section}{Stare Dobre Małżeństwo - Wiersz Księżycowy}
\rightline{(Stare Dobre Małżeństwo-"Krzysztof Myszkowski")} 
\vspace{0.5cm}
W księżycowy wniknąć chłód, 		\tab    {\color{red}\textbf{G a h C} } \newline 
Wejść w to srebro na wskroś złote, 	\tab    {\color{red}\textbf{G D e} } \newline 
W niezawiły śmierci cud 			\tab    {\color{red}\textbf{G a h C} } \newline 
I w zawiłą beztęsknotę! 			\tab    {\color{red}\textbf{G D C} } \newline 

Był tam niegdyś czar i śmiech,	\newline 
Tłumy bogów w snów obłędzie		\newline 
Było dwóch i było trzech		\newline 
Lecz żadnego już nie będzie!	\newline 

Został po nich – rozpęd wzwyż,	\tab    {\color{red}\textbf{G C} } \newline 
I ta oddal bez przyczyny,		\tab    {\color{red}\textbf{a h} } \newline 
I ten złoty nadmiar cisz,		\tab    {\color{red}\textbf{G C} } \newline 
I te srebrne szumowiny...		\tab    {\color{red}\textbf{a h e} } \newline 

{\color{red}\textbf{G a h C G D e G a h C G D C} } \newline

Tam bym ciebie spotkać chciał!		\newline 
Tam się przyjrzeć twemu licu!		\newline 
Właśnie dwojga naszych ciał		\newline 
Brak mi teraz na księżycu!		\newline 

Noc oddycha naszą krwią,		\newline 
Krew podziemną płynie miedzą...	\newline 
Nasze ciała teraz śpią 			\newline 
Nasze ciała nic nie wiedzą...	\newline 

\end{document}
