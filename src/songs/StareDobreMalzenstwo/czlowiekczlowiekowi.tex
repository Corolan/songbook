\documentclass[../../../songbook.tex]{subfiles}
\begin{document}
\TabPositions{8cm} %indywidualnie dla piosenki
\section*{CZŁOWIEK CZŁOWIEKOWI}
\addcontentsline{toc}{section}{Stare Dobre Małżeństwo - Człowiek Człowiekowi}
\rightline{(Stare Dobre Małżeństwo-"Pod Wielkim Dachem Nieba")} 
\vspace{0.5cm}

Człowiek człowiekowi wilkiem 	\tab    {\color{red}\textbf{g d} } \newline
Człowiek człowiekowi strykiem 	\tab    {\color{red}\textbf{F g} } \newline
Lecz ty się nie daj zgnębić		\newline
Lecz ty się nie daj spętlić		\newline

\-\hspace{1cm} Człowiek człowiekowi szpadą	\newline
\-\hspace{1cm} Człowiek człowiekowi zdradą	\newline
\-\hspace{1cm} Lecz ty się nie daj zgładzić	\newline
\-\hspace{1cm} Lecz ty się nie daj zdradzić	\newline

Człowiek człowiekowi pumą 		\tab    {\color{red}\textbf{a e} } \newline
Człowiek człowiekowi dżumą 		\tab    {\color{red}\textbf{G a} } \newline
Lecz ty się nie daj pumie		\newline
Lecz ty się nie daj dżumie		\newline

\-\hspace{1cm} Człowiek człowiekowi łomem	\newline
\-\hspace{1cm} Człowiek człowiekowi gromem	\newline
\-\hspace{1cm} Lecz ty się nie daj zgłuszyć	\newline
\-\hspace{1cm} Lecz ty się nie daj skruszyć	\newline

Człowiek człowiekowi wilkiem		\tab    {\color{red}\textbf{h fis} } \newline
Lecz ty się nie daj zwilczyć 		\tab    {\color{red}\textbf{A h} } \newline
Człowiek człowiekowi bliźnim 		\tab    {\color{red}\textbf{h fis} } \newline
Z bliźnim się możesz zabliźnić 		\tab    {\color{red}\textbf{A h A h} } \newline
\end{document}
