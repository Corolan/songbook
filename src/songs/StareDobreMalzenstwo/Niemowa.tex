\documentclass[../../../songbook.tex]{subfiles}
\begin{document}
\TabPositions{8cm} %indywidualnie dla piosenki
\section*{NIEMOWA}
\phantomsection  
\addcontentsline{toc}{section}{Stare Dobre Małżeństwo - Niemowa}
\rightline{(Stare Dobre Małżeństwo-"Miejska Strona Księżyca")} 
\vspace{0.5cm}
\-\hspace{1cm} Wagony słów! Węglarki słów!			\tab    {\color{red}\textbf{E7} } \newline
\-\hspace{1cm} Prawdziwe złoto, diamenty, perły, \newline
\-\hspace{1cm} Bardzo niewiele sztucznej biżuterii!	\tab    {\color{red}\textbf{A7  E7} } \newline
\-\hspace{1cm} Miłości bujnej, cudnej kuźnie		\tab    {\color{red}\textbf{H7 A7} } \newline
\-\hspace{1cm} I wszystko w próżnię,				\tab    {\color{red}\textbf{E7} } \newline
\-\hspace{1cm} W próżnię, wszystko w próżnię!		\tab    {\color{red}\textbf{H7 E7} } \newline

Nie mówię, nie otwieram ust,	\tab    {\color{red}\textbf{H7 A7} } \newline
Bo już mi niewymownie żal		\tab    {\color{red}\textbf{E7 H7} } \newline
Moich do ciebie słów			\tab    {\color{red}\textbf{E7} } \newline

\-\hspace{1cm} Wagony... \newline

Nie mówię, nie otworzę ust,		\tab    {\color{red}\textbf{H7 A7} } \newline
Żeby już nie było mi żal		\tab    {\color{red}\textbf{E7 H7} } \newline
Moich do ciebie słów			\tab    {\color{red}\textbf{E7} } \newline

{\color{red}\textbf{E7 A7 E7 H7 A7  E7 H7 E7} } \newline

\-\hspace{1cm} Wagony... \newline

Nie mówię, nie otworzę ust,	 \newline
Żeby już nie było mi żal	 \newline	
Moich do ciebie słów	 \newline	
Moich do ciebie słów	 \newline	

Anieli przyszli, \newline
Zagrali w ciszy, \newline
Nikt ich nie słyszy, nie słyszy, \newline
Nikt ich nie słyszy \newline

\end{document}
