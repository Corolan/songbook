\documentclass[../../../songbook.tex]{subfiles}
\begin{document}
\TabPositions{8cm} %indywidualnie dla piosenki
\section*{UWIELBIAJ DUSZO MOJA PANA}
\phantomsection  
\addcontentsline{toc}{section}{Religijne - Uwielbiaj Duszo Moja Pana}
\rightline{(Religijne-"Nieznany")} 
\vspace{0.5cm}
Uwielbiaj duszo moja sławę Pana mego,				    \tab    {\color{red}\textbf{E H cis E} } \newline	  
Adoramus Te Domine										\tab    {\color{red}\textbf{A  H  E/H} } \newline
Chwal Boga Stworzyciela tak bardzo dobrego.				 \newline	
Adoramus Te Domine				 \newline	

Bóg mój, zbawienie moje, jedyna otucha, 				 \newline	
Adoramus Te Domine				 \newline	
Bóg mi rozkoszą serca i weselem ducha.				 \newline	
Adoramus Te Domine				 \newline	

\-\hspace{1cm} Aaa, aaa, aaa,				 \newline	
\-\hspace{1cm} Adoramus Te Domine 				 \newline	

Bo mile przyjąć raczył swej sługi pokorę, 				 \newline	
łaskawym okiem wejrzał na Dawida córę.				 \newline	
Przeto wszystkie narody, co ziemię osiędą, 				 \newline	
odtąd błogosławioną mnie nazywać będą.				 \newline	

Bo wielkimi darami uczczonam od Tego, 				 \newline	
którego moc przedziwna, święte Imię Jego.				 \newline	
Którzy się Pana boją, szczęśliwi na wieki, 				 \newline	
bo z nimi miłosierdzie z rodu w ród daleki.				 \newline	

Na cały świat pokazał moc swych ramion świętych, 				 \newline	
rozproszył dumne myśli głów pychą nadętych.				 \newline	
Wyniosłych złożył z tronu, znikczemnił wielmożne, 				 \newline	
wywyższył, uwielmożnił w pokorę zamożne.				 \newline	

Głodnych nasycił hojnie i w dobra spanoszył, 				 \newline	
bogaczów z niczym puścił i nędznie rozproszył.				 \newline	
Przyjął do łaski sługę, Izraela cnego, 				 \newline	
wspomniał nań, użyczył mu miłosierdzia swego.				 \newline	

Wypełnił, co był przyrzekł niegdyś ojcom naszym, 				 \newline	
Abrahamowi z potomstwem jego wiecznym czasem.				 \newline	
Wszyscy śpiewajmy Bogu w Trójcy Jedynemu, 				 \newline	
chwała Ojcu, Synowi, Duchowi Świętemu.				 \newline	
Jak była na początku, tak zawsze niech będzie, 				 \newline	
teraz i na wiek wieków niechaj słynie wszędzie.				 \newline	

\end{document}
