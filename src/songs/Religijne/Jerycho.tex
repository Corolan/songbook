\documentclass[../../../songbook.tex]{subfiles}
\begin{document}
\TabPositions{8cm} %indywidualnie dla piosenki
\section*{JERYCHO}
\phantomsection  
\addcontentsline{toc}{section}{Religijne - Jerycho}
\rightline{(Religijne-"Nieznany")} 
\vspace{0.5cm}
Ta droga prowadzi do Jerycha,			\tab    {\color{red}\textbf{g c} } \newline
nią Boży zbliża się przechodzień		\tab    {\color{red}\textbf{D g} } \newline
Lecz co to głos jakiś tęskny słychać,	\tab    {\color{red}\textbf{g c} } \newline
to ślepiec woła przy tej drodze			\tab    {\color{red}\textbf{D g} } \newline
Pan zatrzymał się przy nim				\tab    {\color{red}\textbf{D } } \newline
co mam ci uczynić odpowiedz 			\tab    {\color{red}\textbf{g} } \newline
co ci potrzeba							\tab    {\color{red}\textbf{F} } \newline

\-\hspace{1cm} Synu Dawida daj wzrok,			\tab    {\color{red}\textbf{g c} } \newline
\-\hspace{1cm} Boży przechodniu daj wzrok		\tab    {\color{red}\textbf{F B} } \newline
\-\hspace{1cm} Panie, nie widzę od lat,			\tab    {\color{red}\textbf{G c} } \newline
\-\hspace{1cm} Chcę przejrzeć choćby na chwilę	\tab    {\color{red}\textbf{D g} } \newline

Tą drogą Pan także lubi wędrować,  \newline
bo chce byś Boże widział światy \newline
Wystarczą znów wiary pełne słowa,  \newline
byś przejrzał jak ślepiec sprzed laty \newline
Pan na pewno przystanie, gdy usłyszy wołanie, \newline
zapyta co ci potrzeba.\newline

\-\hspace{1cm} Synu Dawida... \newline
\end{document}
