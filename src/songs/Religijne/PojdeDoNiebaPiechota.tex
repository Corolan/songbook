\documentclass[../../../songbook.tex]{subfiles}
\begin{document}
\TabPositions{8cm} %indywidualnie dla piosenki
\section*{PÓJDĘ DO NIEBA PIECHOTĄ}
\phantomsection  
\addcontentsline{toc}{section}{Religijne - Pójdę Do Nieba Piechotą}
\rightline{(Religijne-"Nieznany")} 
\vspace{0.5cm}
Jak jest po tamtej stronie snu,		\tab    {\color{red}\textbf{G} } \newline
czy duchy widzą Wielki Wóz,			\tab    {\color{red}\textbf{a} } \newline
czy duchy widzą Ziemi kształt,		\tab    {\color{red}\textbf{C} } \newline
jak znoszą szybkość świetlnych lat.	\tab    {\color{red}\textbf{D G} } \newline

Czy rozróżnią biel i czerń, \newline
dobro od zła, od nocy dzień, \newline
czy między sobą żrą się jak psy, \newline
czy toczą wojny tak jak my. \newline

\-\hspace{1cm} Pójdę do nieba piechotą,		\tab    {\color{red}\textbf{G a} } \newline
\-\hspace{1cm} przez wodę, przez błoto,		\tab    {\color{red}\textbf{C} } \newline
\-\hspace{1cm} nie pytaj mnie po co?		\tab    {\color{red}\textbf{G/D} } \newline
\-\hspace{1cm} Pójdę ze Stróżem Aniołem,	\tab    {\color{red}\textbf{G a} } \newline
\-\hspace{1cm} tak pójdę jak stoję,			\tab    {\color{red}\textbf{C} } \newline
\-\hspace{1cm} nie zabiorę z sobą nic.		\tab    {\color{red}\textbf{D G} } \newline

Na powitanie wielki bal, \newline
duchy tańczące w tysiąc par. \newline
Anioł wodzirej daje znak, \newline
Bóg ze wszystkimi za pan brat. \newline

Wnet Anioł Stróż przedstawił mnie, \newline
duchom, aniołom, Bogu też. \newline
Znasz może Polskę, biedny kraj, \newline
oto jej syn, rękę mu daj. \newline 

\-\hspace{1cm} Pójdę... \newline

\end{document}
